\documentclass[12pt]{report}
\usepackage{amsmath}
\usepackage{graphicx}
\usepackage{hyperref}
\usepackage[utf8]{inputenc}
\usepackage{listings}

\title{CS3031 Advanced Telecommunications \\ Project I}
\author{Séamus Woods \\ 15317173}
\date{19/02/2019}

\begin{document}
\maketitle
\newpage


\section{Specification}
The objective of the excercise is to implement a Web Proxy Server. A Web proxy is a local server, which fetches items from the Web on behalf of a Web client instead of the client fetching them directly. This allows for caching of pages and access control.
\newline
\newline
The program should be able to:
\begin{itemize}
\item[1.] Respond to HTTP \& HTTPS requests, and should display each request on a management console. It should forward the request to the Web server and relay the response to the browser.
\item[2.] Handle websocket connections.
\item[3.] Dynamically block selected URLs via the management console.
\item[4.] Efficiently cache requests locally and thus save bandwidth. You must gather timing and bandwidth data to prove the efficiency of your proxy.
\item[5.] Handle multiple requests simultaneously by implementing a threaded server. 
\end{itemize}

\section{HTTP \& HTTPS Requests}


\section{Websocket Connections}


\section{Dynamically Blockings URLs}


\section{Caching}


\section{Threading}


\section{Code}
\lstset{%
  language=Python,
  basicstyle=\footnotesize,
  showstringspaces=false,
  numbers=left,
  breakatwhitespace=false,
  breaklines=true,
  breakatwhitespace=true,
}
\lstinputlisting[language=Python]
{proxy.py}

\end{document}